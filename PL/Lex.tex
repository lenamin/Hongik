\documentclass{article}

\usepackage{amsmath,amssymb}
\usepackage{kotex}
\usepackage{graphicx}
\usepackage{subfigure}
\usepackage{caption}
\usepackage{verbatim}

\title{프로그래밍언어 HW2}
\author{B911404 석혜민}

\listfiles
\begin{document}
\maketitle

\section{실행결과}
\includegraphics[width=100mm]{result.png}

\section{프로그램 동작방식}
\subsection{Lexical Analyzer Generator}
Lexical Analyzer Generator는 text의 어휘패턴을 인식하는 프로그램으로, 일정한 구조를 가진 입력을 의미있는 단위{\footnotesize (Unit)}로 분해하여 관련성을 파악할 수 있게 도와주는 프로그램이다.\\ \\
입력을 토큰으로 나누고, 식별할 수 있는 루틴을 생성하는 스캐너이다.

\subsection{구조}
Lex는 세 가지 구조. 즉, 정의절, 규칙절, 사용자 서브루틴절로 구성된다. \\ \\

\subsubsection{Definition Section 정의절}
- 정의절은 \%\{ \%\}로 감싸져있으며 최종 프로그램에 삽입하려는 C 프로그램에 필요한 코드를 작성한다. \\ \\
- header와 필요한 변수 등을 정의할 수 있다.

\begin{quote}
\begin{verbatim}
%{

#include <stdio.h>
// 필요한 변수들 선언 및 초기화
int count = 0;

%}
\end{verbatim}
\end{quote}

\subsubsection{Rules Section 규칙절}
- 규칙절에서는 들어오는 input stream에 대해 적용할 규칙을 정의한다. 패턴과 연산 형태로 이루어지는데, 패턴은 정규표현식으로, 연산은 C언어 형태로 처리가 가능하다. \\ \\
- 패턴을 작성할 때는 위에서 아래로 순차적으로 적용되므로, 다수의 패턴에서 중복된 부분이 있다면 구성요소가 더 많은 쪽부터 매칭을 해야 오류가 나지 않는다. \\ \\
- 또한 default 값은 echo를 반환하므로, 명시적인 처리를 해줘야 한다. \\ \\
- 패턴절의 시작과 끝에는 \%\%을 표기한다. 
\\
\begin{quote}
\begin{verbatim}
%%
[0-9]+	{count++;}
// 10진수 숫자가 1개 이상 있을 때 변수 count를 1씩 더하라는 의미
%%
\end{verbatim}
\end{quote}

\subsubsection{User Subroutine Section 사용자 서브루틴절}
- 사용자가 서브루틴을 적용하는 부분이다. 
- 해당 부분은 입력하는 c 파일에 그대로 복사된다. 
\\
\begin{quote}
\begin{verbatim}
int main() {
    yylex() // 스캐너 함수 
    // input stream과 만났을 때 수행할 연산들 작성
    return 0;
}

int yywrap() { // yylex()가 EOF를 만나면 호출되는 함수
    return 1;
}

\end{verbatim}
\end{quote}

\section{코드}
\subsection{hw2.l}
\begin{normalsize}
\textbf{Definition Section 정의절} \\
\end{normalsize}
정의절에서는 필요한 헤더와 사용될 변수를 초기화해주었다. \\
\begin{quote}
\begin{verbatim}
%{
#include <stdio.h>
int preprosessor_count = 0;    // 전처리 개수
int octal_number_count = 0;    // 8진수 개수
int negative_number_count = 0; // 음수 개수
int positive_number_count = 0; // 양수 개수
int operator_count = 0;        // 연산자 개수
int comment_count = 0;         // 주석 개수
int equal_count = 0;           // 등호 개수
int open_brace_count = 0;      // '{' 개수
int close_brace_count = 0;     // '}' 개수
int word_case_count1 = 0;      // p가 두 번만 들어간 단어의 개수
int word_case_count2 = 0;      // e로 시작해 m으로 끝나는 단어의 개수
int word_count = 0;            // 그 외 단어의 개수
int mark_count = 0;            // 위에서 count 되지 않은 문자의 개수
%}

\end{verbatim}
\end{quote}

\noindent
\begin{normalsize}
\textbf{Rule Section 규칙절} \\
\end{normalsize}
규칙절에서는 정규표현식을 이용해 패턴을 매칭하기 위한 패턴을 작성하고, 그에 따른 연산을 C언어로 작성했다. \\

\begin{quote}
\begin{verbatim}
%%
#(.*)\n    {preprosessor_count++;} 
      /* 전처리기 개수를 찾기 위한 패턴. 
         #으로 시작하는 문장이 있으면
         preprosessor_count 변수 값이 1씩 증가한다  */


0[0-7]+    {octal_number_count++;}
    // 8진수 숫자를 찾기 위한 패턴. 


-[1-9]+    {negative_number_count++;}
    /* 음수를 찾기 위한 패턴 
       양수를 먼저 매칭할 경우, 음수에 있는 숫자도 양수로 인식하므로 
       음수를 먼저 매칭하여 양수를 쉽게 찾을 수 있도록 하였다  */


[1-9][0-9]*    {positive_number_count++;}
    // 양수를 찾으면 positive_number_count값을 1 증가한다 


"//"(.*)\n    {comment_count++;}
    /* 주석 // 을 찾기 위한 패턴 
       연산자인 '/'보다 먼저 찾아야 매칭 오류가 나지 않으므로 
       '//'주석을 먼저 매칭하였다. 
       해당 패턴은 //뒤에 어떤 문자가 오든 //으로 시작하는 문자열은 
       모두 찾아 comment_count를 1씩 증가시킨다.  */


"/*"([^*]|(\*+[^*/]))*"*/"    {comment_count++;}
    /* '/*' '*/' 주석을 찾고, 주석 안에 있는 문자열은 모두 
       이후 패턴에서 찾지 못하도록 하는 패턴 
       시작과 끝은 '/*', '*/'으로 이루어져 있고,
       [^*] : *이 아닌 문자
       (\*+[^*/]) : *가 하나 이상 있고 이어서 
       */가 나오지 않는 문자열을 매칭한다. */


\+{2}|\-{2}    {operator_count++;}
    /* ++, -- 연산자를 매칭하는 패턴
       +, -보다 먼저 매칭해야 ++, --의 개수를 정확하게 찾고, 
       남은 문자들 중에서 +, -를 찾을 수 있으므로 
       ++, -- 연산자를 먼저 매칭했다. 
    */

  
[a-zA-Z]\*    {operator_count++;}
    /* 포인터연산자를 매칭하기 위한 패턴 
       앞에 어떤 문자와 함께 *이 오면 포인터로 매칭한다. 
        *보다 먼저 포인터연산자를 매칭해 정확하게 개수를 카운트할 수 있도록 하였다.
    */

  
\+|\-|\*|\/|%    {operator_count++;}
    // 산술연산자를 매칭하기 위한 패턴 


!    {operator_count++;}
    // 논리연산자 !를 찾으면 operator_count가 1씩 증가한다 
  
  
&{2}    {operator_count++;}
\|{2}    {operator_count++;}
    /*  &&, || 연산자를 찾는다.
        {}안에 반복할 횟수를 넣어 반복할 수 있도록 작성하였다. 
        메타기호 | 와 구별하기 위해 \ 를 추가하였다. */
  
  
={2}	{operator_count++;} 
!=      {operator_count++;}
    // ==, != 연산자를 찾으면 operator_count를 1 증가시킨다 
  
  
>=    {operator_count++;}
\<=    {operator_count++;}
>|\<	{operator_count++;}
    /* >=, <=, >, < 연산자를 찾으면 operator_count를 1 증가시킨다. 
       대소관계인 >, < 보다 먼저 매칭을 해야 >=, <=를 먼저 찾고, 
       이후에 >, <를 찾을 수 있기에 순서에 유의하였다. */


,    {operator_count++;}
&    {operator_count++;}
    // , & 연산자를 찾으면 operator_count를 1 증가시킨다 
  
  
=    {equal_count++;}
     /* = 연산자를 찾으면 equal_count를 1 증가시킨다. 
        <=, >= 연산자는 앞서 미리 찾았기 때문에 =의 개수만 카운트할 수 있다. */

  
\{    {open_brace_count++;}
\}    {close_brace_count++;}
    /* {, } 연산자를 찾으면 각각의 변수에 1씩 더한다. 
       반복을 하도록 하는 메타기호와 헷갈리지 않기 위해 \ 를 붙였다. */

  
p{2}	{word_case_count1++;}
    /* p가 두 번 나오면 word_case_count1 변수가 1 증가한다. 
       /*, */ 주석 내부는 이미 처리하였으므로, 
       주석을 제외한 pp의 개수만 카운트한다   */

  
e[a-zA-Z]m    {word_case_count2++;}
    /* e로 시작하고 m으로 끝나는 문자를 찾아
       word_case_count 변수를 1 증가시킨다 
       [a-zA-Z]가 있어 e와 m사이에 어떤 문자가 있든 
       e로 시작해 m으로 끝나는 문자를 찾을 수 있다   */

  
[0-9a-zA-Z_]*    {word_count++;}
    // 위에서 찾지 않은 그 외 단어의 개수를 카운트한다. 


\n    {mark_count++;}
.     {mark_count++;}
    // 카운트되지 않았던 남은 문자의 개수를 모두 카운트한다.
%%

\end{verbatim}
\end{quote}

\noindent
\begin{normalsize}
\textbf{User Subroutine 사용자 서브루틴절} \\
\end{normalsize}
규칙절에서 카운트한 변수들을 요구사항에 맞게 출력하는 C 코드를 작성하였다.  \\
\begin{quote}
\begin{verbatim}

int main() {
    yylex(); // 스캐너 함수 
    printf("preprocessor = %d \n", preprosessor_count); 
    // 전처리문 개수 -> 2 출력
    printf("octal number = %d \n", octal_number_count); 
    // 8진법 수 -> 1 출력
    printf("negative decimal number = %d \n", negative_number_count); 
    // 10진법 수 중 음수 -> 1 출력
    printf("positive decimal number = %d \n", positive_number_count); 
    // 10진법 수 중 양수 -> 2 출력
    printf("operator = %d \n", operator_count); 
    // 연산자 개수 -> 4 출력
    printf("comment = %d \n", comment_count); 
    // 주석의 개수 -> 2 출력
    printf("'=' = %d \n", equal_count); 
    // '=' 개수 -> 4 출력
    printf("'{' = %d \n", open_brace_count); 
    // '{' 개수 -> 1 출력
    printf("'}' = %d \n", close_brace_count); 
    // '}' 개수 -> 1 출력
    printf("wordcase1 = %d \n", word_case_count1); 
    // pp 개수 -> 2 출력
    printf("wordcase2 = %d \n", word_case_count2); 
    // e로 시작해 m으로 끝나는 문자의 개수 -> 1 출력
    printf("word = %d \n", word_count); 
    // 그 외 카운트되지 않은 문자의 개수 -> 13 출력
    printf("mark = %d \n", mark_count); 
    // 나머지 카운트 되지 않은 개수 -> 37 출력 
    return 0;
}

// yywrap()은 yylex()가 EOF를 만나면 호출한다 
int yywrap() {
    return 1;
}

\end{verbatim}
\end{quote}

\subsection{Trouble Shooting}
- lex에서 지원하지 않는 정규표현식을 사용한 경우 (negative lookbehind와 negative lookahead 등) bad character 오류가 나 lex가 컴파일 할 수 있는 정규표현식으로 수정하였다. \\ \\
\noindent
- 한 번 같은 패턴을 매칭하게 되면, 다음 매칭 때 같은 패턴이 나오는 경우는 warning rule cannot be matched 라는 오류가 뜬다. 이 오류로 인해 한 번 찾았던 동일한 패턴을 매칭하게 되면 오류가 나는 것을 알아, 문제를 해결할 수 있었다. \\

\end{document}
	
